\documentclass[12pt,letterpaper]{article}
\usepackage{natbib}

%Packages
% \usepackage{textcomp}
% \usepackage{latexsym}
% \usepackage{url}
% \usepackage{amssymb}
% \usepackage{amsmath}
% \usepackage{mathtools}
% \usepackage{bm}
% \usepackage{array}
% \usepackage[version=3]{mhchem}
% \usepackage{ifthen}
% \usepackage{amsthm}
% \usepackage{amstext}
% \usepackage{enumerate}
% \usepackage{dcolumn}

\usepackage{epsfig}
\usepackage{graphicx}
\usepackage{caption}
\usepackage{hyperref}
\usepackage{lineno}
\usepackage{pdflscape}
\usepackage{mathtools}
\usepackage[osf]{mathpazo}
\usepackage{fullpage}
\usepackage{float}
\usepackage{xr} %linking to supplementaries
\externaldocument{supplementaries}

\pagenumbering{arabic}


%---------------------------------------------
%
%       START
%
%---------------------------------------------

\begin{document}
%Running head
\begin{flushright}
Version dated: \today
\end{flushright}

\bigskip
\bigskip
\begin{center}

\noindent{\Large \bf The what, how and why of trait-based analyses in ecology}
\bigskip

\noindent {\normalsize \sc
Thomas Guillerme$^{1,*}$, 
Pedro Cardoso$^{2,3}$,
% Carlos P. Carmona$^{4}$,
Maria Wagner J\o rgensen$^{4}$,
Stefano Mammola$^{3,5,6}$,
Thomas J. Matthews$^{4,7}$}\\
\noindent {\small \it 
$^1$School of Biosciences, The University of Sheffield, Sheffield, S10 2TN, United Kingdom.\\
$^2$CE3C—Centre for Ecology, Evolution and Environmental Changes, CHANGE – Global Change and Sustainability Institute, Faculty of Sciences, University of Lisbon, Lisbon, Portugal\\
$^3$Laboratory for Integrative Biodiversity Research (LIBRe), Finnish Museum of Natural History (Luomus), University of Helsinki, Helsinki, Finland\\
% $^4$Institute of Ecology and Earth Sciences, University of Tartu, Tartu, Estonia\\
$^4$School of Geography, Earth and Environmental Sciences and Birmingham Institute of Forest Research, University of Birmingham, Birmingham B15 2TT, UK\\
$^5$Molecular Ecology Group (MEG), Water Research Institute, National Research Council (CNR-IRSA), Verbania Pallanza, Italy\\
$^6$National Biodiversity Future Center, Piazza Marina 61, 90133 Palermo, Italy\\
$^7$Centre for Ecology, Evolution and Environmental Changes/Azorean Biodiversity Group / CHANGE – Global Change and Sustainability Institute and Universidade dos Açores – Faculty of Agricultural Sciences and Environment; PT-9700-042, Angra do Hero\'ismo, A\c ores, Portugal.\\}

\end{center}
\bigskip
\noindent{*\bf Corresponding author.} \textit{t.guillerme@sheffield.ac.uk}\\ 
\vspace{1in}

%Line numbering
\modulolinenumbers[1]
\linenumbers

%---------------------------------------------
%
%       ABSTRACT
%
%---------------------------------------------


\section{Supplementary materials}

\subsection{Glossary}


\begin{table}[ht]
\caption{Glossary and equivalence of terms used in this papers; adapted from \cite{guillerme2018disprity} and \cite{mammola2021concepts}.}
\centering
\begin{tabular}{p{2.5cm}p{4cm}p{5cm}p{5cm}}
Term in this paper & definition & in ecology & in macroevolution \\
\hline
Trait space & Matrix ($n \times d$) with a structural relation between rows and columns & Functional space, morphospace , etc. & Morphospace, traitspace, etc.  \\
\hline
Observations & Rows ($n$) & Taxa, field sites, environments, etc. & Taxa, specimen, populations, elements, OTUs etc. \\
\hline
Dimensions & Columns ($d$) & Traits, Ordination scores, distances, etc. & Traits, ordination scores, distances, etc. \\
\hline
Traits & Columns subset ($b \times n$; $b \leq d$) & uni/multidimensional trait; e.g. geographical location & uni/multidimensional trait; e.g. landmark coordinates \\
\hline
Group & Rows subset ($m \times d$; $m \leq n$) & Treatments, phylogenetic group (clade), etc. & Clades, geological stratum, etc. \\
\hline
Metric & Statistic (i.e. a measure) & Dissimilarity index or metric, hypervolume, functional diversity, etc. & Disparity metric or index \\
\hline
Stressors & An algorithm removing a proportion of the observations & Introduction of invasive species, change of landscape use, climate change, pollution, experimental design, etc. & Mass extinction, Tectonic change, climate change, etc. \\
\hline
Stressors' intensity & The amount of observation removed (percentages) & Proportion of species affected, proportion of temperature change, etc. & Severity of mass extinction,  etc. \\
\end{tabular}
\end{table}



\subsection{Supplementary results}

\begin{figure}[!htbp]
\centering
   \includegraphics[width=0.8\textwidth]{Figures/results_per_metric_per_stressor_2d.pdf}
\caption{\scriptsize{\textbf{Simulation results for 2 dimensions:} the y axes represent the different metrics tested (sorted by categories).
The different columns represent the different stressors. The x-axes represent the metric values centred on the random changes and scaled by the maximum value for each metric between the four stressors.
Negative and positive values signify a decrease/increase in the metric score.
The dots represent the median metric value, the full line their 50\% confidence interval (CI) and the dashed line their 95\% CI.
The colours are here to visually separate the metrics by categories (blue = regularity, yellow = divergence, green = richness); the colour gradient within each row corresponds to a removal of respectively 80\%, 60\%, 40\% and 20\% of the data (from top to bottom).
The grey line plots represent distributions of metric scores not clearly distinguishable from the random metric scores (paired t-test p value $> 0.05$).
Grey lines in the background across the distributions of different removal amounts represent the fitted linear model centred and scaled metric score $\sim$ amount of data removed and the value displayed is the adjusted $R^2$ from each of these models.
Dashed thin grey lines represent non-significant models (p value of slope or/and intercept $> 0.05$).
}}
\label{Fig:simulation_results}
\end{figure}
\bigskip


\begin{figure}[!htbp]
\centering
   \includegraphics[width=0.8\textwidth]{Figures/results_per_metric_per_stressor_8d.pdf}
\caption{\scriptsize{\textbf{Simulation results for 8 dimensions:} the y axes represent the different metrics tested (sorted by categories).
The different columns represent the different stressors. The x-axes represent the metric values centred on the random changes and scaled by the maximum value for each metric between the four stressors.
Negative and positive values signify a decrease/increase in the metric score.
The dots represent the median metric value, the full line their 50\% confidence interval (CI) and the dashed line their 95\% CI.
The colours are here to visually separate the metrics by categories (blue = regularity, yellow = divergence, green = richness); the colour gradient within each row corresponds to a removal of respectively 80\%, 60\%, 40\% and 20\% of the data (from top to bottom).
The grey line plots represent distributions of metric scores not clearly distinguishable from the random metric scores (paired t-test p value $> 0.05$).
Grey lines in the background across the distributions of different removal amounts represent the fitted linear model centred and scaled metric score $\sim$ amount of data removed and the value displayed is the adjusted $R^2$ from each of these models.
Dashed thin grey lines represent non-significant models (p value of slope or/and intercept $> 0.05$).
}}
\label{Fig:simulation_results}
\end{figure}
\bigskip

\begin{landscape}
\begin{table}
\scriptsize
\caption{Results of the model scaled metric $\sim$ removal level per stressor (2D)}
\centering
\begin{tabular}[t]{l|c|c|c|c|c|c|c|c}
\hline
  & equalizing slope & equalizing adj.$R^{2}$ & facilitation slope & facilitation adj.$R^{2}$ & filtering slope & filtering adj.$R^{2}$ & competition slope & competition adj.$R^{2}$\\
\hline
Functional evenness\\distances\\ \texttt{FD::dbFD()\$FEve} & 0.017*** & 0.02 & -0.015** & 0.012 & 0.026*** & 0.057 & -0.001 & -0.001\\
\hline
Functional evenness\\dendrogram\\ \texttt{BAT::evenness} & 0.014** & 0.011 & -0.016** & 0.014 & 0.024*** & 0.038 & -0.009* & 0.007\\
\hline
Functional evenness\\hypervolume\\ \texttt{BAT::kernel.evenness} & 0.007 & 0.002 & -0.006 & 0 & 0.03*** & 0.069 & 0.004 & 0\\
\hline
Functional evenness\\probability density\\ \texttt{TPD::REND} & 0.002 & -0.001 & -0.055*** & 0.07 & 0.001 & -0.001 & -0.02*** & 0.048\\
\hline
Rao's quadratic entropy\\distances\\ \texttt{FD::dbFD()\$RaoQ} & 0.185*** & 0.578 & 0.022** & 0.009 & 0.102*** & 0.117 & 0.022** & 0.013\\
\hline
Functional dispersion\\dendrogram\\ \texttt{BAT::dispersion} & 0.08*** & 0.396 & 0.009. & 0.003 & 0.054*** & 0.195 & 0.008* & 0.005\\
\hline
Functional dispersion\\hypervolume\\ \texttt{BAT::kernel.dispersion} & 0.018** & 0.009 & -0.023*** & 0.04 & 0.014* & 0.004 & -0.022*** & 0.048\\
\hline
Divergence\\probability density\\ \texttt{TPD::REND} & 0.066*** & 0.436 & -0.028*** & 0.046 & -0.009. & 0.004 & -0.022*** & 0.038\\
\hline
Alpha diversity\\dendrogram\\ \texttt{BAT::alpha} & 0.227*** & 0.572 & 0.044*** & 0.037 & 0.091*** & 0.155 & 0.002 & -0.001\\
\hline
Functional richness\\convex hull\\ \texttt{BAT::hull.alpha} & -0.09*** & 0.293 & -0.002 & -0.001 & -0.027*** & 0.043 & -0.011* & 0.008\\
\hline
Functional richness\\hypervolume\\ \texttt{BAT::kernel.alpha} & 0.018*** & 0.049 & 0.046*** & 0.179 & 0.004 & 0 & -0.012*** & 0.02\\
\hline
Functional richness\\probability density\\ \texttt{TPD::REND} & 0.09*** & 0.356 & 0.14*** & 0.28 & 0.022*** & 0.037 & -0.009* & 0.005\\
\hline
\end{tabular}
\end{table}
\end{landscape}



\begin{landscape}
\begin{table}
\scriptsize
\caption{Results of the model scaled metric $\sim$ removal level per stressor (4D)}
\centering
\begin{tabular}[t]{l|c|c|c|c|c|c|c|c}
\hline
  & equalizing slope & equalizing adj.$R^{2}$ & facilitation slope & facilitation adj.$R^{2}$ & filtering slope & filtering adj.$R^{2}$ & competition slope & competition adj.$R^{2}$\\
\hline
Functional evenness\\distances\\ \texttt{FD::dbFD()\$FEve} & 0.054*** & 0.143 & -0.009 & 0.002 & 0.032*** & 0.076 & 0.002 & 0\\
\hline
Functional evenness\\dendrogram\\ \texttt{BAT::evenness} & 0.013* & 0.007 & -0.032*** & 0.092 & 0.015** & 0.012 & -0.002 & 0\\
\hline
Functional evenness\\hypervolume\\ \texttt{BAT::kernel.evenness} & 0.049*** & 0.122 & 0.008 & 0.002 & 0.061*** & 0.194 & 0.018*** & 0.038\\
\hline
Functional evenness\\probability\\ \texttt{TPD::REND} & 0.03*** & 0.135 & -0.01 & 0.002 & 0.024*** & 0.09 & 0.001 & -0.001\\
\hline
Rao's quadratic entropy\\distances\\ \texttt{FD::dbFD()\$RaoQ} & 0.164*** & 0.539 & -0.02** & 0.012 & 0.039*** & 0.023 & -0.002 & -0.001\\
\hline
Functional dispersion\\dendrogram\\ \texttt{BAT::dispersion} & 0.108*** & 0.535 & -0.009** & 0.009 & 0.067*** & 0.258 & 0.012*** & 0.025\\
\hline
Functional dispersion\\hypervolume\\ \texttt{BAT::kernel.dispersion} & 0.064*** & 0.145 & -0.004 & 0.002 & 0.045*** & 0.05 & 0.005* & 0.004\\
\hline
Divergence\\probability density\\ \texttt{TPD::REND} & 0.11*** & 0.52 & -0.03*** & 0.071 & 0.034*** & 0.065 & 0.004 & 0.001\\
\hline
Alpha diversity\\dendrogram\\ \texttt{BAT::alpha} & 0.153*** & 0.393 & 0.059*** & 0.083 & 0.014* & 0.005 & -0.035*** & 0.048\\
\hline
Functional richness\\convex hull\\ \texttt{BAT::hull.alpha} & -0.118*** & 0.517 & 0.013*** & 0.018 & -0.025*** & 0.06 & -0.002 & 0\\
\hline
Functional richness\\hypervolume\\ \texttt{BAT::kernel.alpha} & -0.008** & 0.009 & -0.015*** & 0.016 & -0.006 & 0.002 & -0.001 & -0.001\\
\hline
Functional richness\\probability density\\ \texttt{TPD::REND} & 0.016*** & 0.037 & 0.196*** & 0.535 & 0.014*** & 0.02 & -0.001 & -0.001\\
\hline
\end{tabular}
\end{table}
\end{landscape}



\begin{landscape}
\begin{table}
\scriptsize
\caption{Results of the model scaled metric $\sim$ removal level per stressor (8D)}
\centering
\begin{tabular}[t]{l|c|c|c|c|c|c|c|c}
\hline
  & equalizing slope & equalizing adj.$R^{2}$ & facilitation slope & facilitation adj.$R^{2}$ & filtering slope & filtering adj.$R^{2}$ & competition slope & competition adj.$R^{2}$\\
\hline
Functional evenness\\distances\\ \texttt{FD::dbFD()\$FEve} & 0.039*** & 0.071 & 0.013** & 0.009 & 0.026*** & 0.051 & 0 & -0.001\\
\hline
Functional evenness\\dendrogram\\ \texttt{BAT::evenness} & -0.023*** & 0.041 & -0.038*** & 0.122 & -0.024*** & 0.102 & -0.008*** & 0.056\\
\hline
Functional evenness\\hypervolume\\ \texttt{BAT::kernel.evenness} & -0.055*** & 0.159 & -0.051*** & 0.121 & -0.034*** & 0.093 & -0.007*** & 0.015\\
\hline
Functional evenness\\probability\\ \texttt{TPD::REND} & -0.032*** & 0.134 & -0.118*** & 0.345 & -0.059*** & 0.185 & -0.008*** & 0.03\\
\hline
Rao's quadratic entropy\\distances\\ \texttt{FD::dbFD()\$RaoQ} & 0.088*** & 0.157 & -0.02** & 0.01 & 0.013 & 0.002 & -0.012. & 0.004\\
\hline
Functional dispersion\\dendrogram\\ \texttt{BAT::dispersion} & 0.05*** & 0.166 & -0.022*** & 0.055 & 0.01. & 0.003 & -0.005 & 0.002\\
\hline
Functional dispersion\\hypervolume\\ \texttt{BAT::kernel.dispersion} & 0.035*** & 0.035 & -0.004 & 0.001 & 0.028*** & 0.029 & -0.007** & 0.009\\
\hline
Divergence\\probability density\\ \texttt{TPD::REND} & 0.037*** & 0.121 & -0.029*** & 0.101 & -0.005 & 0 & -0.017*** & 0.041\\
\hline
Alpha diversity\\dendrogram\\ \texttt{BAT::alpha} & 0.03*** & 0.019 & 0.08*** & 0.131 & 0.041*** & 0.028 & -0.041*** & 0.05\\
\hline
Functional richness\\convex hull\\ \texttt{BAT::hull.alpha} & -0.077*** & 0.265 & 0.031*** & 0.09 & -0.009* & 0.006 & 0.002 & -0.001\\
\hline
Functional richness\\hypervolume\\ \texttt{BAT::kernel.alpha} & -0.019** & 0.013 & -0.023*** & 0.017 & -0.02*** & 0.018 & 0.002 & -0.001\\
\hline
Functional richness\\probability density\\ \texttt{TPD::REND} & 0.014** & 0.012 & 0.175*** & 0.574 & 0.08*** & 0.112 & -0.002 & -0.001\\
\hline
\end{tabular}
\end{table}
\end{landscape}



\bibliographystyle{sysbio}
\bibliography{references}


\end{document}