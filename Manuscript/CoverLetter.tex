\documentclass[11pt]{letter}
\usepackage[a4paper,left=2.5cm, right=2.5cm, top=1cm, bottom=1cm]{geometry}
\usepackage{hyperref}
\usepackage[osf]{mathpazo}
\signature{Thomas Guillerme}
\address{School of Biosciences\\University of Sheffield\\Sheffield, S10 2TN\\United Kingdom\\t.guillerme@sheffield.ac.uk}
\longindentation=0pt
\begin{document}

\begin{letter}{}
\opening{Dear Editors,}

Since the seminal publication by Petchey and Gaston in Ecology Letters in 2002 on functional diversity (``Functional diversity (FD), species richness and community composition''; cumulating $>$ 2000 citations), functional diversity is now routinely used as a statistic to describe complex multidimensional datasets.
This has been driven by a really motivated community of ecologists proposing many different ways to measure functional diversity depending on different datasets or different questions.
More than 20 years later however, despite the enthusiasm of the ecology and evolution community there is no consensual way on which metric to use and in which scenarios they perform better.
This has led to the positive and rich paradigm of a diverse number of metrics routinely used for different tasks.

However, this richness of ways to measure functional diversity can make it daunting and or confusing for researchers that want to add functional diversity to their research methodology toolkit.
We argue that this confusions can stem from both an epistemological misunderstanding as well as practical hurdles.
The second is nicely alleviated by a dynamic community of researchers proposing open source and easy to use implementations to measure functional diveristy.
But the first hurdle remains.
Indeed, although it is often rather clear how functional diversity in general can capture changes communities (investigating processes; answering the question ``how are communities changing?'') and relate them to general understanding in ecology (understanding mechanisms; ``why are communities changing?''); surprisingly little work has been published in the last few years on how the metrics used to measure functional diversity can accurately measure complex multidimensional patterns (capturing patterns; ``what is changing?'').

Here we argue that a better understanding of the ``capturability'' of a pattern should be taken more into account for functional diversity to move forward as a great tool in ecology and evolution.
Through simulated and empirical data we show that the choice of the functional diversity metric (what) can lead to often very different interpretations of the same processes and mechanisms (how and why).
For example, in our empirical data looking at endemic birds extinctions in Hawaii we show that different functional diversity metrics can lead to widely different interpretations of the actual biological mechanisms changing Hawaiian bird communities.
I.e. with the same dataset and a drastic rate of extinction for endemic birds, different metrics can either show a clear decrease in functional diversity (expected) but also a clear increase (unexpected).
We have written this manuscript with an emphasis on the didactic side of the question and we hope it will be of great interest and usefulness to the many readers of Ecology Letters using functional diversity in their research.

I look forward to hearing from you soon.

\closing{Yours sincerely,\\On behalf of my co-authors}

\end{letter}
\end{document}