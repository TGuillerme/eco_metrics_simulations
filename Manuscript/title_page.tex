\documentclass[12pt,letterpaper]{article}
\usepackage{natbib}

%Packages
% \usepackage{textcomp}
% \usepackage{latexsym}
% \usepackage{url}
% \usepackage{amssymb}
% \usepackage{amsmath}
% \usepackage{mathtools}
% \usepackage{bm}
% \usepackage{array}
% \usepackage[version=3]{mhchem}
% \usepackage{ifthen}
% \usepackage{amsthm}
% \usepackage{amstext}
% \usepackage{enumerate}
% \usepackage{dcolumn}

\usepackage{epsfig}
\usepackage{graphicx}
\usepackage{caption}
\usepackage{hyperref}
\usepackage{lineno}
\usepackage{pdflscape}
\usepackage{mathtools}
\usepackage[osf]{mathpazo}
\usepackage{fullpage}
\usepackage{float}
\usepackage{xr} %linking to supplementaries
\externaldocument{supplementaries}

\pagenumbering{arabic}

%---------------------------------------------
%
%       START
%
%---------------------------------------------

\begin{document}
%Running head
\begin{flushright}
Version dated: \today
\end{flushright}

\bigskip
\begin{center}

\noindent{\Large \bf The what, how and why of trait-based analyses in ecology}
\bigskip

\noindent RH: The what, how and why of trait-based analyses

\noindent {\normalsize \sc
Thomas Guillerme$^{1,*}$, 
Pedro Cardoso$^{2,3}$, %pedro.cardoso@helsinki.fi
% Carlos P. Carmona$^{4}$,
Maria Wagner J\o rgensen$^{4}$, %mwj207@student.bham.ac.uk
Stefano Mammola$^{3,5,6}$, %stefanomammola@gmail.com
Thomas J. Matthews$^{4,7}$}\\ % txm676@gmail.com
\noindent {\small \it 
$^1$School of Biosciences, The University of Sheffield, Sheffield, S10 2TN, United Kingdom.\\
$^2$CE3C—Centre for Ecology, Evolution and Environmental Changes, CHANGE – Global Change and Sustainability Institute, Faculty of Sciences, University of Lisbon, Lisbon, Portugal\\
$^3$Laboratory for Integrative Biodiversity Research (LIBRe), Finnish Museum of Natural History (Luomus), University of Helsinki, Helsinki, Finland\\
% $^4$Institute of Ecology and Earth Sciences, University of Tartu, Tartu, Estonia\\
$^4$School of Geography, Earth and Environmental Sciences and Birmingham Institute of Forest Research, University of Birmingham, Birmingham B15 2TT, UK\\
$^5$Molecular Ecology Group (MEG), Water Research Institute, National Research Council (CNR-IRSA), Verbania Pallanza, Italy\\
$^6$National Biodiversity Future Center, Piazza Marina 61, 90133 Palermo, Italy\\
$^7$Centre for Ecology, Evolution and Environmental Changes/Azorean Biodiversity Group / CHANGE – Global Change and Sustainability Institute and Universidade dos Açores – Faculty of Agricultural Sciences and Environment; PT-9700-042, Angra do Hero\'ismo, A\c ores, Portugal.\\}

\end{center}
\bigskip
\noindent{*\bf Corresponding author.} \textit{t.guillerme@sheffield.ac.uk}\\ 

\section{Acknowledgements}
Thanks to Andrew Beckerman, Natalie Cooper, Alain Danet, Thomas Johnson and Gavin Thomas for comments on early versions of this manuscripts.
TG was funded by the UKRI-NERC grant NE/X016781/1.
SM was supported by NBFC, funded by the Italian Ministry of University and Research, PNRR, Missione 4, Componente 2, ``Dalla ricerca all'impresa'', Investimento 1.4, Project CN00000033.
MWJ was supported by NERC CENTA2 grant NE/S007350/1 and University of Birmingham.

\subsection{Conflict of interest}
We declare no conflict of interest.

\subsection{Authors contributions}
TG, PC, MWJ, SM and TJM designed the study, analysed the data and wrote the manuscript. TG wrote the code to analyse the data.

\subsection{Data accessibility statement}
The simulations, results figures and tables are entirely reproducible via \url{https://github.com/TGuillerme/eco_metrics_simulations}.
An additional vignette on how to choose metrics is available at \url{https://github.com/TGuillerme/eco_metrics_simulations/blob/master/Analysis/Choosing_metrics_vignette.Rmd}.




\end{document}